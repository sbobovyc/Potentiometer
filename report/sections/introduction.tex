\section{Introduction}
The motivation behind this project is to get familiar with MEMS sensors. Specifically, how to
manufacture them, interface them to embedded systems, profile and characterize them. The
construction of a portable glucose sensor is a good fit for this task. The construction of this system
incurs low cost per unit and cross cuts across the disciplines of chemistry and computer engineering.

There are major challenges presented in this project. Manufacturing the sensor is a laborious process
with many steps and many places where things could go wrong. Also, our sensor will not be ready for
profiling until the end of the semester, so we will have to make some assumptions about how we tune
our amplification circuitry for certain current and voltage ranges. Profiling the sensor will take time
and we will be working with unfamiliar lab equipment. In addition to hardware, we will need to write
desktop and embedded software. Finally, debugging the pieces of the system and then integrating the
various modules will be a challenge.

Our approach to building a glucose sensor is to use a potentiostat in voltammetry mode, so that we can
measure concentration of a glucose solution by the amount of current flowing between two electrodes.
The current measurements will be streamed to a PC via an XBee radio, where the data will be post
processed and displayed in a user friendly graph format.

The expected result is that our system will be robust, low power, user friendly and that we will be able
to detect at least 3 different concentrations of glucose.

\cite{Gopinath:2006}