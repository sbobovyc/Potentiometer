\section{Conclusion}

This project describes a detailed implementation of an inexpensive, and portable potentiostat unit and the process of manufacturing and testing of glucose sensors. Unfortunately, it is not possible to directly compare this system with a commercial one since we were not able to characterize our own sensors and did not have time to test commercial ones. We would like to make the observation that the custom strip could be redesigned with a divot on the active area to help keep the hydrogel from spreading all over the strip.

However, the system presented can measure the electrochemical currents between the rages of 0.1uA to 70uA and is suitable to a wide variety of chemical detection applications. The system is also low power, field portable and can be used for remote monitoring in real time. We also provide a schematic and footprint, so it is ready to be mass produced. 

The system is very robust and the PC software we developed to interface with the system is user friendly. The output voltage ranges (potential) can be easily altered by changing the feedback resistance of the opamp. Thus, the design's gain is not limited and any reaction that has the electron transfer process is detectable with this system. Further, a digital potentiometer or a bank of resistors with a multi-pole switch in place of the feedback resistor can make gain adjustment programmatic and even automatic. Given more time, we would like to have fabricated a PCB version of our system.

Continuous glucose monitoring can be applied to variety of situations including management of diabetes as well as treatment of trauma patients and others with temporarily elevated glucose level.  Furthermore, this study confirms previous findings in potentiostat design.