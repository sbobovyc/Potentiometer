\section{Background}

Measuring glucose has been around for a long time. \cite{Cramp:1967}
 Salenization, platenization, potentiostats, etc

Immobilizing oxidase enzymes. Oxygen is the electron acceptor. Measure $ H_{2}O_{2}$ produced during enzymatic reaction. Oxidation produces current. Platenizing electrodes increases surface area and increases catalytic activity. The diamond pattern provides a larger surface area and helps convert $ H_{2}O_{2}$ to an electrical current. 
\cite{Zhang:1996}

Silenization and derivatization uses covalent bonding to immobilize oxidase. \cite{Urban:1991}

Immobilizing and enzyme attaches it to an inert, insoluble material. This makes the sensor tougher, resistant to pH or temperature changes. The enzymes are held in place during the chemical reaction. Stability of the reaction and consistency between reactions is ensured. It also increases shelf life of sensor. Many different techniques exist. Adsorption, covalent binding, affinity immobilization, entrapment along with many different organic and inorganic materials. \cite{Datta:2013} 

When glucose and glucose oxidaze are combined in the presense of water and oxygen, gluconic acid and hydrogen peroxide forms. We can then let hydrogen peroxide oxidize into oxygen, hydrogen and electrons. These chemical reactions are described in equations \ref{eq:glux} and \ref{eq:perox}. Electrodes made of manganese dioxide, silver or platinum will help hydrogen peroxide oxidation catalyse. \cite{Raba:1995}

\begin{equation} \label{eq:glux}
\text{glucose} + H_2O + O_2 \xrightarrow{GOx}  \text{gluconic acid} + H_2O_2
\end{equation}

\begin{equation} \label{eq:perox}
H_2O_2 \rightarrow O_2 + 2H^+ + 2e^-
\end{equation}


The current generated by the chemical reaction can be measured. Voltametry, amperometry, potentiostat. 