\section{Background}
A biosensor is a self contained integrated device incorporating a biological material capable of producing an electrical signal. There are many different types of sensors and chemical sensing techniques. Some chemical sensing techniques are potentiometric, amperometric, FET based, conductometric, optical, piezoeletric and thermal. Glucose sensors are one such sensor, and are an important part of medicine due to the rise of diabetes and work on improving them has been a going on for a while. \cite{Cramp:1967} Measuring blood sugar levels is an important part of treating this disease and over the years researchers have come up with a wide range of invasive and non-invasive techniques. \cite{Oliver:2009} For the purpose of this project, we will focus on the amperometric approach using a three-electrode system coupled with the enzyme glucose oxidase which is a very common and non-invasive technique. \cite{Cui:2001} \cite{Jui-Lin:2011} 

\subsection{Basic sensor design}
Chemical sensors and specifically glucose sensors can be manufactured out of a myriad of materials and in different physical layouts. Glucose sensors can be broken into two major categories: invasive and non invasive. We will concentrate on non invasive sensors. Non invasive glucose sensors have gone through several stages of development. \cite{Wang:2012} One of the simpler designs consists of planar layers of substrate and metal with an immobilized enzyme sitting on top.

\subsection{Chemistry}
In order to understand how a glucose oxidase based sensor works, we have to understand the chemistry going on. When glucose and glucose oxidase are combined in the presence of water and oxygen, gluconic acid and hydrogen peroxide forms. After this initial reaction, hydrogen peroxide oxidizes into oxygen, hydrogen and electrons. These chemical reactions are described in equations \ref{eq:glux} and \ref{eq:perox}. The $ H_{2}O_{2}$ produced during enzymatic reaction can be calculated from the current produced during oxidation.

\begin{equation} \label{eq:glux}
\text{glucose} + H_2O + O_2 \xrightarrow{GOx}  \text{gluconic acid} + H_2O_2
\end{equation}

\begin{equation} \label{eq:perox}
H_2O_2 \rightarrow O_2 + 2H^+ + 2e^-
\end{equation}

Design of the sensor has an effect of the speed of this reaction and the durability of the sensor. Using electrodes made of manganese dioxide, silver or platinum will help hydrogen peroxide oxidation catalyse. \cite{Raba:1995}
Another design technique is immobilizing the oxidase enzymes. Immobilizing an enzyme attaches it to an inert, insoluble material. This makes the sensor tougher and resistant to pH or temperature changes. It also increases shelf life of sensor. The enzymes are held in place during the chemical reaction. Stability of the reaction and consistency between reactions is ensured. Many different techniques for immobilization exist such as adsorption, covalent binding, affinity immobilization, entrapment along with many different organic and inorganic materials that promote immobilization \cite{Datta:2013} In our project we use silenization and derivatization for covalent bonding to immobilize oxidase. \cite{Urban:1991} Platinizing the electrodes is another technique. It increases the active surface area and increases catalytic activity. Another way to increase surface area is making the active area with a diamond pattern. Increasing the surface area helps convert $ H_{2}O_{2}$ to an electrical current faster. \cite{Zhang:1996}

\subsection{Potentiostat}
Potentiostat is an instrument that can very accurately measure current between electrodes created by a potential difference between those electrodes. A typical potentiostat will consist of three electrodes: working, reference, and counter. The current flows from the working to the counter electrode. The reference electrode provides a constant electrochemical potential. Having a separate reference electrode helps make current measurements at the working electrode more accurate. The basic design of a potentiostat consists of a controlled voltage supply, a voltmeter and a current to voltage converter. \cite{Penn:2010} Potentiostats can range from expensive bench top lab equipment to inexpensive portable devices. \cite{Row:2011}\cite{Gopinath:2006} In the application of glucose sensing, the potentiostat is used in the amperometry mode to produce a reference voltage. The current is measured at the working electrode and and is a result of the previously described chemical reaction.  This current is directly proportional to the glucose concentration

%The current is used to determine glucose concentration by this equation \ref{eq:gluc}.
%\begin{equation} \label{eq:gluc}
%glucose concentration equation
%\end{equation}
